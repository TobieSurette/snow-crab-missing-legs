\documentclass[letterpaper,10pt]{article}

\usepackage{natbib}
\usepackage{gulf}
\usepackage{graphicx}
\usepackage{pdflscape}
\usepackage{verbatim}
\usepackage{algorithmic}
\usepackage{float}

\newcommand{\R}{\textbf{R~}}

% Specify report title:
\title{Snow Crab Length-Weight Relationship with Missing Legs}
       
% Specify list of authors:     
\author{T. Surette and E. Wade}

% Specify report number:
% \renewcommand{\reportnumber}{XXXX}     

% Specify type of report (i.e. 'technical', 'data', 'manuscript' or 'industry')
\renewcommand{\reporttype}{technical}   

% Specify e-mail address:
\renewcommand{\email}{surettetj@dfo-mpo.gc.ca}  

\begin{document}
\maketitle 

\tableofcontents
\newpage


\section*{Abstract}
\addcontentsline{toc}{section}{Abstract}



\newpage

\section*{List of Tables}
\addcontentsline{toc}{section}{List of Tables}
\renewcommand\listtablename{}
\listoftables
\newpage

\section*{List of Figures}
\addcontentsline{toc}{section}{List of Figures}
\renewcommand\listfigurename{}
\listoffigures
\newpage

\pagenumbering{arabic}

\section{Introduction}

Snow crab loose chelae or legs through various natural processes such as predation and intra-specific competition. 

An annual snow crab trawl survey is performed in the southern Gulf of Saint Lawrence (Figure X). 

Crab catches are sorted and measured for various characteristics, including size, carapace condition, shell hardness, presence of missing or regenerated legs, but not weighted.

Estimates of trawlable biomass are derived using a size-weight relationship which yields a predicted weight, based on the weight of an intact crab, i.e. a crab with no missing legs.

Owing to the relatively small size of the body, loss of legs can have a significant effect on the overall weight.

Thus the biomass estimates are slightly overestimated as crab found in nature contain a portion which have, through various processes such as predation and intra-specific competition, a number of missing legs.

The present study seeks to derive a length-weight relationship which incorporates the pattern of missing legs in the prediction of individual crab weights.


\section{Methods}


An observation study was performed at two locations.

The first is near Grande Rivière, Québec (n = 2000) and the second is from Chéticamp, Nova Scotia.

Only male snow crab were caught.

Snow crab traps were set for 1-2 days and snow crab were individually measured for size (carapace width), chela height, shell condition, shell hardness, 

Crab with damaged parts, such as the body (carapace) were not included in the analysis.

The observed type of dama

\subsection{Analyses}

\begin{equation}
   w = \alpha x^\beta e^\epsilon 
\end{equation}

Taking the logarithm on each side, this equation is linearized to 

\begin{equation}
   \ln w = \beta \ln x + \ln \alpha + \epsilon
\end{equation}
where $\ln w$ is a linear function of $\ln x$. The exponent $\beta$ becomes the slope parameter and $\ln \alpha$ the intercept parameter.

The model used in the length-weight analyses is an extension of the standard allometric model which allows for variation in the $\alpha$ parameter as a function of the number and position of each missing leg.

\begin{equation}
   \ln w = \beta \ln x + \ln{\left(\alpha_0 + \sum_{i=1}^{5}{\alpha_i m_i} 
                             + \sum_{i=1}^{5}{\gamma_i r_i}\right)} + \epsilon
\end{equation}

where $m_i$ is the number of missing legs and $r_i$ is the number of missing legs at the $i^{th}$ position.

Missing legs may be regenerated during moults. 
Generally, multiple moults are required.  
Thus regenerated legs may be found in various stages of regeneration. 
Confounding of these multiple stages of regeneration will possibly increase the observed predicted variation about the predicted weight.
The coefficient associated with each regenerated leg is thus an averaging of multiple stages of regeneration.

Whether the relationship varies with sample site will be verified. 

As a further analysis, the number of missing legs versus the size of the crab, along with sample site, shell condition.

\section{Results}






\section{Discussion}



\newpage



\end{document}
